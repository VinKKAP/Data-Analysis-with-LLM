% Options for packages loaded elsewhere
\PassOptionsToPackage{unicode}{hyperref}
\PassOptionsToPackage{hyphens}{url}
\PassOptionsToPackage{dvipsnames,svgnames,x11names}{xcolor}
%
\documentclass[
  11pt,
]{article}

\usepackage{amsmath,amssymb}
\usepackage{setspace}
\usepackage{iftex}
\ifPDFTeX
  \usepackage[T1]{fontenc}
  \usepackage[utf8]{inputenc}
  \usepackage{textcomp} % provide euro and other symbols
\else % if luatex or xetex
  \usepackage{unicode-math}
  \defaultfontfeatures{Scale=MatchLowercase}
  \defaultfontfeatures[\rmfamily]{Ligatures=TeX,Scale=1}
\fi
\usepackage{lmodern}
\ifPDFTeX\else  
    % xetex/luatex font selection
\fi
% Use upquote if available, for straight quotes in verbatim environments
\IfFileExists{upquote.sty}{\usepackage{upquote}}{}
\IfFileExists{microtype.sty}{% use microtype if available
  \usepackage[]{microtype}
  \UseMicrotypeSet[protrusion]{basicmath} % disable protrusion for tt fonts
}{}
\makeatletter
\@ifundefined{KOMAClassName}{% if non-KOMA class
  \IfFileExists{parskip.sty}{%
    \usepackage{parskip}
  }{% else
    \setlength{\parindent}{0pt}
    \setlength{\parskip}{6pt plus 2pt minus 1pt}}
}{% if KOMA class
  \KOMAoptions{parskip=half}}
\makeatother
\usepackage{xcolor}
\usepackage[a4paper,left=35mm,right=35mm,top=4cm,bottom=4cm]{geometry}
\setlength{\emergencystretch}{3em} % prevent overfull lines
\setcounter{secnumdepth}{5}
% Make \paragraph and \subparagraph free-standing
\ifx\paragraph\undefined\else
  \let\oldparagraph\paragraph
  \renewcommand{\paragraph}[1]{\oldparagraph{#1}\mbox{}}
\fi
\ifx\subparagraph\undefined\else
  \let\oldsubparagraph\subparagraph
  \renewcommand{\subparagraph}[1]{\oldsubparagraph{#1}\mbox{}}
\fi


\providecommand{\tightlist}{%
  \setlength{\itemsep}{0pt}\setlength{\parskip}{0pt}}\usepackage{longtable,booktabs,array}
\usepackage{calc} % for calculating minipage widths
% Correct order of tables after \paragraph or \subparagraph
\usepackage{etoolbox}
\makeatletter
\patchcmd\longtable{\par}{\if@noskipsec\mbox{}\fi\par}{}{}
\makeatother
% Allow footnotes in longtable head/foot
\IfFileExists{footnotehyper.sty}{\usepackage{footnotehyper}}{\usepackage{footnote}}
\makesavenoteenv{longtable}
\usepackage{graphicx}
\makeatletter
\def\maxwidth{\ifdim\Gin@nat@width>\linewidth\linewidth\else\Gin@nat@width\fi}
\def\maxheight{\ifdim\Gin@nat@height>\textheight\textheight\else\Gin@nat@height\fi}
\makeatother
% Scale images if necessary, so that they will not overflow the page
% margins by default, and it is still possible to overwrite the defaults
% using explicit options in \includegraphics[width, height, ...]{}
\setkeys{Gin}{width=\maxwidth,height=\maxheight,keepaspectratio}
% Set default figure placement to htbp
\makeatletter
\def\fps@figure{htbp}
\makeatother
% definitions for citeproc citations
\NewDocumentCommand\citeproctext{}{}
\NewDocumentCommand\citeproc{mm}{%
  \begingroup\def\citeproctext{#2}\cite{#1}\endgroup}
\makeatletter
 % allow citations to break across lines
 \let\@cite@ofmt\@firstofone
 % avoid brackets around text for \cite:
 \def\@biblabel#1{}
 \def\@cite#1#2{{#1\if@tempswa , #2\fi}}
\makeatother
\newlength{\cslhangindent}
\setlength{\cslhangindent}{1.5em}
\newlength{\csllabelwidth}
\setlength{\csllabelwidth}{3em}
\newenvironment{CSLReferences}[2] % #1 hanging-indent, #2 entry-spacing
 {\begin{list}{}{%
  \setlength{\itemindent}{0pt}
  \setlength{\leftmargin}{0pt}
  \setlength{\parsep}{0pt}
  % turn on hanging indent if param 1 is 1
  \ifodd #1
   \setlength{\leftmargin}{\cslhangindent}
   \setlength{\itemindent}{-1\cslhangindent}
  \fi
  % set entry spacing
  \setlength{\itemsep}{#2\baselineskip}}}
 {\end{list}}
\usepackage{calc}
\newcommand{\CSLBlock}[1]{\hfill\break\parbox[t]{\linewidth}{\strut\ignorespaces#1\strut}}
\newcommand{\CSLLeftMargin}[1]{\parbox[t]{\csllabelwidth}{\strut#1\strut}}
\newcommand{\CSLRightInline}[1]{\parbox[t]{\linewidth - \csllabelwidth}{\strut#1\strut}}
\newcommand{\CSLIndent}[1]{\hspace{\cslhangindent}#1}

\makeatletter
\@ifpackageloaded{caption}{}{\usepackage{caption}}
\AtBeginDocument{%
\ifdefined\contentsname
  \renewcommand*\contentsname{Table of contents}
\else
  \newcommand\contentsname{Table of contents}
\fi
\ifdefined\listfigurename
  \renewcommand*\listfigurename{List of Figures}
\else
  \newcommand\listfigurename{List of Figures}
\fi
\ifdefined\listtablename
  \renewcommand*\listtablename{List of Tables}
\else
  \newcommand\listtablename{List of Tables}
\fi
\ifdefined\figurename
  \renewcommand*\figurename{Figure}
\else
  \newcommand\figurename{Figure}
\fi
\ifdefined\tablename
  \renewcommand*\tablename{Table}
\else
  \newcommand\tablename{Table}
\fi
}
\@ifpackageloaded{float}{}{\usepackage{float}}
\floatstyle{ruled}
\@ifundefined{c@chapter}{\newfloat{codelisting}{h}{lop}}{\newfloat{codelisting}{h}{lop}[chapter]}
\floatname{codelisting}{Listing}
\newcommand*\listoflistings{\listof{codelisting}{List of Listings}}
\makeatother
\makeatletter
\makeatother
\makeatletter
\@ifpackageloaded{caption}{}{\usepackage{caption}}
\@ifpackageloaded{subcaption}{}{\usepackage{subcaption}}
\makeatother
\ifLuaTeX
  \usepackage{selnolig}  % disable illegal ligatures
\fi
\usepackage{bookmark}

\IfFileExists{xurl.sty}{\usepackage{xurl}}{} % add URL line breaks if available
\urlstyle{same} % disable monospaced font for URLs
\hypersetup{
  pdftitle={Seminararbeit AI-assisted programming and data analysis},
  pdfauthor={Vincent-Konstantin Kapp; Prof.~Dr.~Martin Spindler},
  pdfkeywords={LLM, Data Analysis, Data Profiling},
  colorlinks=true,
  linkcolor={blue},
  filecolor={Maroon},
  citecolor={Blue},
  urlcolor={Blue},
  pdfcreator={LaTeX via pandoc}}

\title{Seminararbeit AI-assisted programming and data analysis}
\usepackage{etoolbox}
\makeatletter
\providecommand{\subtitle}[1]{% add subtitle to \maketitle
  \apptocmd{\@title}{\par {\large #1 \par}}{}{}
}
\makeatother
\subtitle{Data Profiling with LLM's - A Case Study on HR Data}
\author{Vincent-Konstantin Kapp \and Prof.~Dr.~Martin Spindler}
\date{31. October 2024}

\begin{document}
\maketitle
\begin{abstract}
Datengetriebene Prozesse können Unternehmen nicht nur dabei helfen, ihre
Prozesse zu optimieren, sondern auch bei der Entwicklung neuer
Strategien. Um festzustellen, welche Daten in den Datenbanken vorliegen
und wie diese zueinander stehen, braucht es Mitarbeiter*innen, die
einzeln überprüfen, welche Daten vorhanden sind und in welchen
Zusammenhängen sie basierend auf Common Sense existieren. Data Profiling
stellt Unternehmen mit historisch gewachsenen Datenbanksystemen vor
Herausforderungen für die datengestützte Transformation. Können die
aktuellen Entwicklungen von Large Language Modellen (LLMs) dazu
beitragen, automatisiert die Zusammenhänge von Spalten zu erkennen? Um
diese Frage zu beantworten, widmet sich diese Seminararbeit, aufbauend
auf der Arbeit von Trummer (2024), der Untersuchung, wie gut LLMs anhand
einer Case Study die Korrelation von Daten erkennen können.
\end{abstract}

\setstretch{1.5}
\section{Einleitung}\label{einleitung}

\section{Hintergrund}\label{hintergrund}

\subsection{Data Profiling}\label{data-profiling}

\subsection{Korrelationserkennung in
Datenbanken}\label{korrelationserkennung-in-datenbanken}

(Trummer 2023).

\subsection{Sprachmodelle und deren Anwendung in der
Datenanalyse}\label{sprachmodelle-und-deren-anwendung-in-der-datenanalyse}

(Arora et al. 2023).

\section{Methodik der Fallstudie}\label{methodik-der-fallstudie}

\subsection{Datenbasis und
Vorbereitung}\label{datenbasis-und-vorbereitung}

\subsection{Vorgehensweise zur Korrelationsanalyse mit
LLM}\label{vorgehensweise-zur-korrelationsanalyse-mit-llm}

\subsection{Benchmark und
Bewertungsmetriken}\label{benchmark-und-bewertungsmetriken}

\section{Durchführung der Fallstudie:
Gender-Pay-Gap-Analyse}\label{durchfuxfchrung-der-fallstudie-gender-pay-gap-analyse}

Meine Überlegung ist es das Ergebnisse von (Bach, Chernozhukov, and
Spindler 2024).

\subsection{Fragestellung}\label{fragestellung}

\subsection{Implementierung des LLM für
Korrelationserkennung}\label{implementierung-des-llm-fuxfcr-korrelationserkennung}

\subsection{Benchmarking-Ergebnisse}\label{benchmarking-ergebnisse}

\section{Vergleich und Auswertung der
Ergebnisse}\label{vergleich-und-auswertung-der-ergebnisse}

\subsection{Vergleich der Ergebnisse mit
Prompoting}\label{vergleich-der-ergebnisse-mit-prompoting}

\subsection{Szenariobasierte
Auswertung}\label{szenariobasierte-auswertung}

\subsection{Bewertung der
Korrelationserkennung}\label{bewertung-der-korrelationserkennung}

\section{Diskussion der Ergebnisse und
Implikationen}\label{diskussion-der-ergebnisse-und-implikationen}

\subsection{Interpretation der
Ergebnisse}\label{interpretation-der-ergebnisse}

\subsection{Limitationen}\label{limitationen}

\subsection{Praktische
Anwendungsmöglichkeit}\label{praktische-anwendungsmuxf6glichkeit}

\section{Fazit und Ausblick}\label{fazit-und-ausblick}

\section*{References}\label{references}
\addcontentsline{toc}{section}{References}

\phantomsection\label{refs}
\begin{CSLReferences}{1}{0}
\bibitem[\citeproctext]{ref-RN5576}
Arora, Simran, Brandon Yang, Sabri Eyuboglu, Avanika Narayan, Andrew
Hojel, Immanuel Trummer, and Christopher Ré. 2023. {``Language Models
Enable Simple Systems for Generating Structured Views of Heterogeneous
Data Lakes.''} Journal Article. \emph{arXiv Preprint arXiv:2304.09433}.

\bibitem[\citeproctext]{ref-bach2024heterogeneity}
Bach, Philipp, Victor Chernozhukov, and Martin Spindler. 2024.
{``Heterogeneity in the US Gender Wage Gap.''} \emph{Journal of the
Royal Statistical Society Series A: Statistics in Society} 187 (1):
209--30.

\bibitem[\citeproctext]{ref-RN5578}
Trummer, Immanuel. 2023. {``Can Large Language Models Predict Data
Correlations from Column Names?''} Journal Article. \emph{Proceedings of
the VLDB Endowment} 16 (13): 4310--23.

\end{CSLReferences}



\end{document}
