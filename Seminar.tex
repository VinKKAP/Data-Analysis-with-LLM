% Options for packages loaded elsewhere
\PassOptionsToPackage{unicode}{hyperref}
\PassOptionsToPackage{hyphens}{url}
\PassOptionsToPackage{dvipsnames,svgnames,x11names}{xcolor}
%
\documentclass[
  11pt,
  authoryear,
  preprint]{elsarticle}

\usepackage{amsmath,amssymb}
\usepackage{setspace}
\usepackage{iftex}
\ifPDFTeX
  \usepackage[T1]{fontenc}
  \usepackage[utf8]{inputenc}
  \usepackage{textcomp} % provide euro and other symbols
\else % if luatex or xetex
  \usepackage{unicode-math}
  \defaultfontfeatures{Scale=MatchLowercase}
  \defaultfontfeatures[\rmfamily]{Ligatures=TeX,Scale=1}
\fi
\usepackage{lmodern}
\ifPDFTeX\else  
    % xetex/luatex font selection
\fi
% Use upquote if available, for straight quotes in verbatim environments
\IfFileExists{upquote.sty}{\usepackage{upquote}}{}
\IfFileExists{microtype.sty}{% use microtype if available
  \usepackage[]{microtype}
  \UseMicrotypeSet[protrusion]{basicmath} % disable protrusion for tt fonts
}{}
\makeatletter
\@ifundefined{KOMAClassName}{% if non-KOMA class
  \IfFileExists{parskip.sty}{%
    \usepackage{parskip}
  }{% else
    \setlength{\parindent}{0pt}
    \setlength{\parskip}{6pt plus 2pt minus 1pt}}
}{% if KOMA class
  \KOMAoptions{parskip=half}}
\makeatother
\usepackage{xcolor}
\usepackage[a4paper,left=35mm,right=35mm,top=4cm,bottom=4cm]{geometry}
\setlength{\emergencystretch}{3em} % prevent overfull lines
\setcounter{secnumdepth}{5}
% Make \paragraph and \subparagraph free-standing
\ifx\paragraph\undefined\else
  \let\oldparagraph\paragraph
  \renewcommand{\paragraph}[1]{\oldparagraph{#1}\mbox{}}
\fi
\ifx\subparagraph\undefined\else
  \let\oldsubparagraph\subparagraph
  \renewcommand{\subparagraph}[1]{\oldsubparagraph{#1}\mbox{}}
\fi


\providecommand{\tightlist}{%
  \setlength{\itemsep}{0pt}\setlength{\parskip}{0pt}}\usepackage{longtable,booktabs,array}
\usepackage{calc} % for calculating minipage widths
% Correct order of tables after \paragraph or \subparagraph
\usepackage{etoolbox}
\makeatletter
\patchcmd\longtable{\par}{\if@noskipsec\mbox{}\fi\par}{}{}
\makeatother
% Allow footnotes in longtable head/foot
\IfFileExists{footnotehyper.sty}{\usepackage{footnotehyper}}{\usepackage{footnote}}
\makesavenoteenv{longtable}
\usepackage{graphicx}
\makeatletter
\def\maxwidth{\ifdim\Gin@nat@width>\linewidth\linewidth\else\Gin@nat@width\fi}
\def\maxheight{\ifdim\Gin@nat@height>\textheight\textheight\else\Gin@nat@height\fi}
\makeatother
% Scale images if necessary, so that they will not overflow the page
% margins by default, and it is still possible to overwrite the defaults
% using explicit options in \includegraphics[width, height, ...]{}
\setkeys{Gin}{width=\maxwidth,height=\maxheight,keepaspectratio}
% Set default figure placement to htbp
\makeatletter
\def\fps@figure{htbp}
\makeatother

\makeatletter
\@ifpackageloaded{caption}{}{\usepackage{caption}}
\AtBeginDocument{%
\ifdefined\contentsname
  \renewcommand*\contentsname{Table of contents}
\else
  \newcommand\contentsname{Table of contents}
\fi
\ifdefined\listfigurename
  \renewcommand*\listfigurename{List of Figures}
\else
  \newcommand\listfigurename{List of Figures}
\fi
\ifdefined\listtablename
  \renewcommand*\listtablename{List of Tables}
\else
  \newcommand\listtablename{List of Tables}
\fi
\ifdefined\figurename
  \renewcommand*\figurename{Figure}
\else
  \newcommand\figurename{Figure}
\fi
\ifdefined\tablename
  \renewcommand*\tablename{Table}
\else
  \newcommand\tablename{Table}
\fi
}
\@ifpackageloaded{float}{}{\usepackage{float}}
\floatstyle{ruled}
\@ifundefined{c@chapter}{\newfloat{codelisting}{h}{lop}}{\newfloat{codelisting}{h}{lop}[chapter]}
\floatname{codelisting}{Listing}
\newcommand*\listoflistings{\listof{codelisting}{List of Listings}}
\makeatother
\makeatletter
\makeatother
\makeatletter
\@ifpackageloaded{caption}{}{\usepackage{caption}}
\@ifpackageloaded{subcaption}{}{\usepackage{subcaption}}
\makeatother
\journal{Journal Name}
\ifLuaTeX
  \usepackage{selnolig}  % disable illegal ligatures
\fi
\usepackage[]{natbib}
\bibliographystyle{elsarticle-harv}
\usepackage{bookmark}

\IfFileExists{xurl.sty}{\usepackage{xurl}}{} % add URL line breaks if available
\urlstyle{same} % disable monospaced font for URLs
\hypersetup{
  pdftitle={Seminararbeit AI-assisted programming and data analysis},
  pdfauthor={Vincent-Konstantin Kapp; Prof.~Dr.~Martin Spindler},
  pdfkeywords={LLM, Data Analysis, Data Profiling},
  colorlinks=true,
  linkcolor={blue},
  filecolor={Maroon},
  citecolor={Blue},
  urlcolor={Blue},
  pdfcreator={LaTeX via pandoc}}

\setlength{\parindent}{6pt}
\begin{document}

\begin{frontmatter}
\title{Seminararbeit AI-assisted programming and data
analysis \\\large{Data Profiling with LLM`s a Case Study on HR Data} }
\author[1]{Vincent-Konstantin Kapp%
%
\fnref{fn1}}
 \ead{vincent.kapp@studium.uni-hamburg.de} 
\author[2]{Prof.~Dr.~Martin Spindler%
%
\fnref{fn2}}
 \ead{martin.spindler@uni-hamburg.de} 

\affiliation[1]{organization={Universität Hamburg, M.Sc.
BWL},addressline={Street Address},city={Hamburg},postcode={Postal
Code},postcodesep={}}
\affiliation[2]{organization={Universität Hamburg, Statistik mit
Anwendung in der Betriebswirtschaftslehre},addressline={Street
Address},city={Hamburg},postcode={Postal Code},postcodesep={}}

\cortext[cor1]{Corresponding author}
\fntext[fn1]{This is the author.}
\fntext[fn2]{This is the Editor}
        
\begin{abstract}
Datengetriebene Prozesse können nicht nur Unternehmen beim Optimieren
der Prozesse sondern auch bei der entwicklung neuer Strategien helfen.
Aber uum festzustellen welche Daten in den Datenbanken vorliegen und wie
diese zu einernander stehen braucht es Mitarbeiter*innen die einzelnd
überprüfen welche Daten es gibt und in welchen zusammenhängen, ausgehend
von Commen Sense, es gibt. Data Profiling stellt Unternehmen mit
historisch gewachsene Datenbanksysteme vor Herausforderungen für die
Dategetriebene Transformation da. Können die aktuellen entwicklunge von
Large Language Modellen (LLM) dazu beitragen, autamtisiert die
zusammenhängen von Spalten zu erkennen? Um diese Frage zu beantworten,
widmet sich diese Seminararbeit aufbauen von der Arbeit von Trummer
2024, wie gut LLM, anhand einer Case Study, die Korrelation von Daten zu
erkennen.
\end{abstract}





\begin{keyword}
    LLM \sep Data Analysis \sep 
    Data Profiling
\end{keyword}
\end{frontmatter}
    
\setstretch{1.5}
Seminararbeit

Philipp Bach, Victor Chernozhukov, Martin Spindler (2024). Heterogeneity
in the U.S. Gender Wage Gap. Journal of the Royal Statistical Society:
Series A, 187(1), 209-230, available online.

Philipp Bach, Victor Chernozhukov, Martin Spindler, Closing the U.S.
gender wage gap requires understanding its heterogeneity, Working Paper,
available at arXiv, 2018.

Sven Klaassen, Jan Teichert-Kluge, Philipp Bach, Victor Chernozhukov,
Martin Spindler, Suhas Vijaykumar, DoubleMLDeep: Estimation of Causal
Effects with Multimodal Data, available at arxiv, 2024.

\citep{RN5574}

\section{Einleitung}\label{einleitung}

\section{Hintergrund}\label{hintergrund}

\subsection{Data Profiling}\label{data-profiling}

\subsection{Korrelation ermittelnt}\label{korrelation-ermittelnt}

\subsection{Sprachmodelle}\label{sprachmodelle}

\subsection{NLP for Data Bases}\label{nlp-for-data-bases}

\subsection{Benchmark}\label{benchmark}

\subsection{Benchmark Data}\label{benchmark-data}

\subsection{Benchnmark Metrics}\label{benchnmark-metrics}

\section{Benchmark Analysis}\label{benchmark-analysis}

\section{Comparing Prediction
Methods}\label{comparing-prediction-methods}

\subsection{Description of Methods}\label{description-of-methods}

\subsection{Experimental Setup}\label{experimental-setup}

\subsection{Comparison Results}\label{comparison-results}

\section{Scenario Variants}\label{scenario-variants}

\section{Results Breakdown}\label{results-breakdown}

\section{Other Correlation Metrics}\label{other-correlation-metrics}

\section{Column Types}\label{column-types}

\section{Conclusion}\label{conclusion}


\renewcommand\refname{References}
  \bibliography{My EndNote Library.bib}


\end{document}
