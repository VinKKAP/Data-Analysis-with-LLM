% Options for packages loaded elsewhere
\PassOptionsToPackage{unicode}{hyperref}
\PassOptionsToPackage{hyphens}{url}
\PassOptionsToPackage{dvipsnames,svgnames,x11names}{xcolor}
%
\documentclass[
  authoryear,
  preprint]{elsarticle}

\usepackage{amsmath,amssymb}
\usepackage{iftex}
\ifPDFTeX
  \usepackage[T1]{fontenc}
  \usepackage[utf8]{inputenc}
  \usepackage{textcomp} % provide euro and other symbols
\else % if luatex or xetex
  \usepackage{unicode-math}
  \defaultfontfeatures{Scale=MatchLowercase}
  \defaultfontfeatures[\rmfamily]{Ligatures=TeX,Scale=1}
\fi
\usepackage{lmodern}
\ifPDFTeX\else  
    % xetex/luatex font selection
\fi
% Use upquote if available, for straight quotes in verbatim environments
\IfFileExists{upquote.sty}{\usepackage{upquote}}{}
\IfFileExists{microtype.sty}{% use microtype if available
  \usepackage[]{microtype}
  \UseMicrotypeSet[protrusion]{basicmath} % disable protrusion for tt fonts
}{}
\makeatletter
\@ifundefined{KOMAClassName}{% if non-KOMA class
  \IfFileExists{parskip.sty}{%
    \usepackage{parskip}
  }{% else
    \setlength{\parindent}{0pt}
    \setlength{\parskip}{6pt plus 2pt minus 1pt}}
}{% if KOMA class
  \KOMAoptions{parskip=half}}
\makeatother
\usepackage{xcolor}
\setlength{\emergencystretch}{3em} % prevent overfull lines
\setcounter{secnumdepth}{5}
% Make \paragraph and \subparagraph free-standing
\ifx\paragraph\undefined\else
  \let\oldparagraph\paragraph
  \renewcommand{\paragraph}[1]{\oldparagraph{#1}\mbox{}}
\fi
\ifx\subparagraph\undefined\else
  \let\oldsubparagraph\subparagraph
  \renewcommand{\subparagraph}[1]{\oldsubparagraph{#1}\mbox{}}
\fi

\usepackage{color}
\usepackage{fancyvrb}
\newcommand{\VerbBar}{|}
\newcommand{\VERB}{\Verb[commandchars=\\\{\}]}
\DefineVerbatimEnvironment{Highlighting}{Verbatim}{commandchars=\\\{\}}
% Add ',fontsize=\small' for more characters per line
\usepackage{framed}
\definecolor{shadecolor}{RGB}{241,243,245}
\newenvironment{Shaded}{\begin{snugshade}}{\end{snugshade}}
\newcommand{\AlertTok}[1]{\textcolor[rgb]{0.68,0.00,0.00}{#1}}
\newcommand{\AnnotationTok}[1]{\textcolor[rgb]{0.37,0.37,0.37}{#1}}
\newcommand{\AttributeTok}[1]{\textcolor[rgb]{0.40,0.45,0.13}{#1}}
\newcommand{\BaseNTok}[1]{\textcolor[rgb]{0.68,0.00,0.00}{#1}}
\newcommand{\BuiltInTok}[1]{\textcolor[rgb]{0.00,0.23,0.31}{#1}}
\newcommand{\CharTok}[1]{\textcolor[rgb]{0.13,0.47,0.30}{#1}}
\newcommand{\CommentTok}[1]{\textcolor[rgb]{0.37,0.37,0.37}{#1}}
\newcommand{\CommentVarTok}[1]{\textcolor[rgb]{0.37,0.37,0.37}{\textit{#1}}}
\newcommand{\ConstantTok}[1]{\textcolor[rgb]{0.56,0.35,0.01}{#1}}
\newcommand{\ControlFlowTok}[1]{\textcolor[rgb]{0.00,0.23,0.31}{#1}}
\newcommand{\DataTypeTok}[1]{\textcolor[rgb]{0.68,0.00,0.00}{#1}}
\newcommand{\DecValTok}[1]{\textcolor[rgb]{0.68,0.00,0.00}{#1}}
\newcommand{\DocumentationTok}[1]{\textcolor[rgb]{0.37,0.37,0.37}{\textit{#1}}}
\newcommand{\ErrorTok}[1]{\textcolor[rgb]{0.68,0.00,0.00}{#1}}
\newcommand{\ExtensionTok}[1]{\textcolor[rgb]{0.00,0.23,0.31}{#1}}
\newcommand{\FloatTok}[1]{\textcolor[rgb]{0.68,0.00,0.00}{#1}}
\newcommand{\FunctionTok}[1]{\textcolor[rgb]{0.28,0.35,0.67}{#1}}
\newcommand{\ImportTok}[1]{\textcolor[rgb]{0.00,0.46,0.62}{#1}}
\newcommand{\InformationTok}[1]{\textcolor[rgb]{0.37,0.37,0.37}{#1}}
\newcommand{\KeywordTok}[1]{\textcolor[rgb]{0.00,0.23,0.31}{#1}}
\newcommand{\NormalTok}[1]{\textcolor[rgb]{0.00,0.23,0.31}{#1}}
\newcommand{\OperatorTok}[1]{\textcolor[rgb]{0.37,0.37,0.37}{#1}}
\newcommand{\OtherTok}[1]{\textcolor[rgb]{0.00,0.23,0.31}{#1}}
\newcommand{\PreprocessorTok}[1]{\textcolor[rgb]{0.68,0.00,0.00}{#1}}
\newcommand{\RegionMarkerTok}[1]{\textcolor[rgb]{0.00,0.23,0.31}{#1}}
\newcommand{\SpecialCharTok}[1]{\textcolor[rgb]{0.37,0.37,0.37}{#1}}
\newcommand{\SpecialStringTok}[1]{\textcolor[rgb]{0.13,0.47,0.30}{#1}}
\newcommand{\StringTok}[1]{\textcolor[rgb]{0.13,0.47,0.30}{#1}}
\newcommand{\VariableTok}[1]{\textcolor[rgb]{0.07,0.07,0.07}{#1}}
\newcommand{\VerbatimStringTok}[1]{\textcolor[rgb]{0.13,0.47,0.30}{#1}}
\newcommand{\WarningTok}[1]{\textcolor[rgb]{0.37,0.37,0.37}{\textit{#1}}}

\providecommand{\tightlist}{%
  \setlength{\itemsep}{0pt}\setlength{\parskip}{0pt}}\usepackage{longtable,booktabs,array}
\usepackage{calc} % for calculating minipage widths
% Correct order of tables after \paragraph or \subparagraph
\usepackage{etoolbox}
\makeatletter
\patchcmd\longtable{\par}{\if@noskipsec\mbox{}\fi\par}{}{}
\makeatother
% Allow footnotes in longtable head/foot
\IfFileExists{footnotehyper.sty}{\usepackage{footnotehyper}}{\usepackage{footnote}}
\makesavenoteenv{longtable}
\usepackage{graphicx}
\makeatletter
\def\maxwidth{\ifdim\Gin@nat@width>\linewidth\linewidth\else\Gin@nat@width\fi}
\def\maxheight{\ifdim\Gin@nat@height>\textheight\textheight\else\Gin@nat@height\fi}
\makeatother
% Scale images if necessary, so that they will not overflow the page
% margins by default, and it is still possible to overwrite the defaults
% using explicit options in \includegraphics[width, height, ...]{}
\setkeys{Gin}{width=\maxwidth,height=\maxheight,keepaspectratio}
% Set default figure placement to htbp
\makeatletter
\def\fps@figure{htbp}
\makeatother

\makeatletter
\@ifpackageloaded{caption}{}{\usepackage{caption}}
\AtBeginDocument{%
\ifdefined\contentsname
  \renewcommand*\contentsname{Table of contents}
\else
  \newcommand\contentsname{Table of contents}
\fi
\ifdefined\listfigurename
  \renewcommand*\listfigurename{List of Figures}
\else
  \newcommand\listfigurename{List of Figures}
\fi
\ifdefined\listtablename
  \renewcommand*\listtablename{List of Tables}
\else
  \newcommand\listtablename{List of Tables}
\fi
\ifdefined\figurename
  \renewcommand*\figurename{Figure}
\else
  \newcommand\figurename{Figure}
\fi
\ifdefined\tablename
  \renewcommand*\tablename{Table}
\else
  \newcommand\tablename{Table}
\fi
}
\@ifpackageloaded{float}{}{\usepackage{float}}
\floatstyle{ruled}
\@ifundefined{c@chapter}{\newfloat{codelisting}{h}{lop}}{\newfloat{codelisting}{h}{lop}[chapter]}
\floatname{codelisting}{Listing}
\newcommand*\listoflistings{\listof{codelisting}{List of Listings}}
\makeatother
\makeatletter
\makeatother
\makeatletter
\@ifpackageloaded{caption}{}{\usepackage{caption}}
\@ifpackageloaded{subcaption}{}{\usepackage{subcaption}}
\makeatother
\journal{Journal Name}
\ifLuaTeX
  \usepackage{selnolig}  % disable illegal ligatures
\fi
\usepackage[]{natbib}
\bibliographystyle{elsarticle-harv}
\usepackage{bookmark}

\IfFileExists{xurl.sty}{\usepackage{xurl}}{} % add URL line breaks if available
\urlstyle{same} % disable monospaced font for URLs
\hypersetup{
  pdftitle={Seminararbeit AI-assisted programming and data analysis},
  pdfauthor={Vincent Kapp; Prof.~Dr.~Martin Spindler},
  pdfkeywords={LLM, Data Analysis, Data Profiling},
  colorlinks=true,
  linkcolor={blue},
  filecolor={Maroon},
  citecolor={Blue},
  urlcolor={Blue},
  pdfcreator={LaTeX via pandoc}}

\setlength{\parindent}{6pt}
\begin{document}

\begin{frontmatter}
\title{Seminararbeit AI-assisted programming and data
analysis \\\large{Data Profiling with LLM`s a Case Study on HR Data} }
\author[1]{Vincent Kapp%
\corref{cor1}%
\fnref{fn1}}
 \ead{vincent.kapp@studium.uni-hamburg.de} 
\author[2]{Prof.~Dr.~Martin Spindler%
%
\fnref{fn2}}
 \ead{martin.spindler@uni-hamburg.de} 

\affiliation[1]{organization={Universität Hamburg, M.Sc.
BWL},addressline={Street Address},city={Hamburg},postcode={Postal
Code},postcodesep={}}
\affiliation[2]{organization={Universität Hamburg, Statistik mit
Anwendung in der Betriebswirtschaftslehre},addressline={Street
Address},city={Hamburg},postcode={Postal Code},postcodesep={}}

\cortext[cor1]{Corresponding author}
\fntext[fn1]{This is the first author footnote.}
\fntext[fn2]{Another author footnote, this is a very long footnote and
it should be a really long footnote. But this footnote is not yet
sufficiently long enough to make two lines of footnote text.}
        
\begin{abstract}
This is the abstract. Lorem ipsum dolor sit amet, consectetur adipiscing
elit. Vestibulum augue turpis, dictum non malesuada a, volutpat eget
velit. Nam placerat turpis purus, eu tristique ex tincidunt et. Mauris
sed augue eget turpis ultrices tincidunt. Sed et mi in leo porta
egestas. Aliquam non laoreet velit. Nunc quis ex vitae eros aliquet
auctor nec ac libero. Duis laoreet sapien eu mi luctus, in bibendum leo
molestie. Sed hendrerit diam diam, ac dapibus nisl volutpat vitae.
Aliquam bibendum varius libero, eu efficitur justo rutrum at. Sed at
tempus elit.
\end{abstract}





\begin{keyword}
    LLM \sep Data Analysis \sep 
    Data Profiling
\end{keyword}
\end{frontmatter}
    
Please make sure that your manuscript follows the guidelines in the
Guide for Authors of the relevant journal. It is not necessary to
typeset your manuscript in exactly the same way as an article, unless
you are submitting to a camera-ready copy (CRC) journal.

For detailed instructions regarding the elsevier article class, see
\url{https://www.elsevier.com/authors/policies-and-guidelines/latex-instructions}

\section{Einleitung}\label{einleitung}

Datengetriebene Prozesse können nicht nur Unternehmen beim Optimieren
der Prozesse sondern auch bei der entwicklung neuer Strategien helfen.
Aber uum festzustellen welche Daten in den Datenbanken vorliegen und wie
diese zu einernander stehen braucht es Mitarbeiter*innen die einzelnd
überprüfen welche Daten es gibt und in welchen zusammenhängen, ausgehend
von Commen Sense, es gibt. Data Profiling stellt Unternehmen mit
historisch gewachsene Datenbanksysteme vor Herausforderungen für die
Dategetriebene Transformation da. Können die aktuellen entwicklunge von
Large Language Modellen (LLM) dazu beitragen, autamtisiert die
zusammenhängen von Spalten zu erkennen? Um diese Frage zu beantworten,
widmet sich diese Seminararbeit aufbauen von der Arbeit von Trummer
2024, wie gut LLM, anhand einer Case Study, die Korrelation von Daten zu
erkennen.

Philipp Bach, Victor Chernozhukov, Martin Spindler (2024). Heterogeneity
in the U.S. Gender Wage Gap. Journal of the Royal Statistical Society:
Series A, 187(1), 209-230, available online.

Philipp Bach, Victor Chernozhukov, Martin Spindler, Closing the U.S.
gender wage gap requires understanding its heterogeneity, Working Paper,
available at arXiv, 2018.

Sven Klaassen, Jan Teichert-Kluge, Philipp Bach, Victor Chernozhukov,
Martin Spindler, Suhas Vijaykumar, DoubleMLDeep: Estimation of Causal
Effects with Multimodal Data, available at arxiv, 2024.

\citet{RN5574}

Here are two sample references: \citet{Feynman1963118}
\citet{Dirac1953888}.

With this template using elsevier class, natbib will be used. Three
bibliographic style files (*.bst) are provided and their use controled by
\texttt{cite-style} option:

\begin{itemize}
\tightlist
\item
  \texttt{citestyle:\ number} (default) will use
  \texttt{elsarticle-num.bst} - can be used for the numbered scheme
\item
  \texttt{citestyle:\ numbername} will use
  \texttt{elsarticle-num-names.bst} - can be used for numbered with new
  options of natbib.sty
\item
  \texttt{citestyle:\ authoryear} will use \texttt{elsarticle-harv.bst}
  --- can be used for author year scheme
\end{itemize}

This \texttt{citestyle} will insert the right \texttt{.bst} and set the
correct \texttt{classoption} for \texttt{elsarticle} document class.

Using \texttt{natbiboptions} variable in YAML header, you can set more
options for \texttt{natbib} itself . Example

\begin{Shaded}
\begin{Highlighting}[]
\FunctionTok{natbiboptions}\KeywordTok{:}\AttributeTok{ longnamesfirst,angle,semicolon}
\end{Highlighting}
\end{Shaded}

\section{Hintergrund}\label{hintergrund}

If \texttt{cite-method} is set to \texttt{citeproc} in
\texttt{elsevier\_article()}, then pandoc is used for citations instead
of \texttt{natbib}. In this case, the \texttt{csl} option is used to
format the references. By default, this template will provide an
appropriate style, but alternative \texttt{csl} files are available from
\url{https://www.zotero.org/styles?q=elsevier}. These can be downloaded
and stored locally, or the url can be used as in the example header.

\subsection{Data Profiling}\label{data-profiling}

\subsection{Korrelation ermittelnt}\label{korrelation-ermittelnt}

\subsection{Sprachmodelle}\label{sprachmodelle}

\subsection{NLP for Data Bases}\label{nlp-for-data-bases}

\subsection{Benchmark}\label{benchmark}

\subsubsection{Equations}\label{equations}

Here is an equation: \[ 
  f_{X}(x) = \left(\frac{\alpha}{\beta}\right)
  \left(\frac{x}{\beta}\right)^{\alpha-1}
  e^{-\left(\frac{x}{\beta}\right)^{\alpha}}; 
  \alpha,\beta,x > 0 .
\]

Inline equations work as well: \(\sum_{i = 2}^\infty\{\alpha_i^\beta\}\)

\subsubsection{Figures and tables}\label{figures-and-tables}

\textbf{?@fig-meaningless} is generated using an R chunk.

\#\texttt{\{r\}\ \#\textbar{}\ label:\ fig-meaningless\ \#\textbar{}\ fig-cap:\ A\ meaningless\ scatterplot\ \#\textbar{}\ fig-width:\ 5\ \#\textbar{}\ fig-height:\ 5\ \#\textbar{}\ fig-align:\ center\ \#\textbar{}\ out-width:\ 50\%\ \#\textbar{}\ echo:\ false\ plot(runif(25),\ runif(25))}

\section{Tables coming from R}\label{tables-coming-from-r}

Tables can also be generated using R chunks, as shown in
\textbf{?@tbl-simple} example.

\#\texttt{\{r\}\ \#\textbar{}\ label:\ tbl-simple\ \#\textbar{}\ tbl-cap:\ Caption\ centered\ above\ table\ \#\textbar{}\ echo:\ true\ knitr::kable(head(mtcars){[},1:4{]})}

\subsection{Benchmark Data}\label{benchmark-data}

\subsection{Benchnmark Metrics}\label{benchnmark-metrics}

\section{Benchmark Analysis}\label{benchmark-analysis}

\section{Comparing Prediction
Methods}\label{comparing-prediction-methods}

\subsection{Description of Methods}\label{description-of-methods}

\subsection{Experimental Setup}\label{experimental-setup}

\subsection{Comparison Results}\label{comparison-results}

\section{Scenario Variants}\label{scenario-variants}

\section{Results Breakdown}\label{results-breakdown}

\section{Other Correlation Metrics}\label{other-correlation-metrics}

\section{Column Types}\label{column-types}

\section{Conclusion}\label{conclusion}


\renewcommand\refname{References}
  \bibliography{My EndNote Library.bib}


\end{document}
